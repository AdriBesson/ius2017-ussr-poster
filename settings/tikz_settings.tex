% *** TIKZ PACKAGES ***
\usepackage{xparse}
\usepackage{tikz}
\usetikzlibrary{shapes, arrows.meta, decorations.pathmorphing}
\usetikzlibrary{calc, positioning}
\usetikzlibrary{backgrounds}
\usetikzlibrary{intersections} % provides name path=
\usetikzlibrary{angles} % provides pic angle

\usepackage{etoolbox}

% Tikzset
\tikzset{%
	block/.style = {draw, shape=rectangle, thick, minimum height=2em, minimum width=2em},
	sum/.style = {draw, shape=circle, thick, inner sep=0pt},
	conn/.style={fill, shape=circle, minimum size=0.1cm, inner sep=0, outer sep=0},
	inout/.style={inner sep=0, outer sep=0},
	dot/.style={fill=, shape=circle, minimum size=\pgflinewidth, inner sep=0, outer sep=0},
	param/.style={inner sep=0, outer sep=1pt},
	loosely dotted round/.style={dash pattern=on 0pt off 6\pgflinewidth, line cap=round},
	ultra thin/.style= {line width=0.1pt},
	very thin/.style=  {line width=0.2pt},
	thin/.style=       {line width=0.4pt},% thin is the default
	semithick/.style=  {line width=0.6pt},
	thick/.style=      {line width=3pt},
	very thick/.style= {line width=1.2pt},
	ultra thick/.style={line width=1.6pt}
	%	transducer/.style = {shape=rectangle, draw=black, fill=black!20!white, thick, minimum height=0.15cm, minimum width=0.30cm, inner sep=0pt},
	%	my funny rectangle/.style n args={4}{%
	%    rectangle,
	%    draw,
	%    fit={(#1,#3) (#2,#4)}
	%%    append after command={\pgfextra{\let\mainnode=\tikzlastnode}
	%%      node[above right] at (\mainnode.north west) {#3}%
	%%      node[above left] at (\mainnode.north east) {#4}%
	%%      node[below left] at (\mainnode.north west) {#1}%
	%%      node[above left] at (\mainnode.south west) {#2}%
	%%    },
	%  }
}

%\def\td_number{3}

%\newcommand*{\transducer}[4][-1][-2]{\path[draw=black, fill=black!20!white] (#3, #1) rectangle (#4, #2)}

% arg1: lx (optional, default=1)
% arg2: ly (optional, default=2)
% arg3: x_center
% arg4: y_center
%\NewDocumentCommand{\transducer}{ O{-1} O{-2} m }{\path[draw=black, fill=black!20!white] (#3, #1) rectangle (#4, #2)}

\NewDocumentCommand{\rectangle}{ m m }{%
	%\newlength{\Xone}{#1 cm}%\pgfmathparse{#1+1}\pgfmathresult
	\path[draw=black, fill=black!20!white] #1 rectangle #2
}

%\pgfdeclarelayer{bg}    % declare background layer
%\pgfsetlayers{bg,main}  % set the order of the layers (main is the standard layer)

% This bascially automates a \newcommand{<name>}{} to ensure
% that a command with the given <name> does not already exist
\newcommand*{\pgfmathsetnewmacro}[2]{%
	\newcommand*{#1}{}% Error if already defined
	\pgfmathsetmacro{#1}{#2}%
}%

% Transducer commands
\newcommand*{\transducerNumb}{10}
\newcommand*{\transducerWidth}{0.75}
\newcommand*{\transducerHeight}{0.25}
\newcommand*{\transducerInitXPos}{3}
\newcommand*{\transducerXOffset}{1}
\definecolor{transducer-color}{gray}{0.8}
\newcommand*{\transducerLabeled}{9}
\newcommand*{\transducerLabeledBis}{7}

% Measurement commands
\newcommand*{\measurementLength}{6}
\definecolor{measurement-color}{gray}{0.7}

% Insonified domain
\pgfmathsetmacro{\domainXMin}{\transducerInitXPos}
\pgfmathsetmacro{\domainXMax}{12}
\pgfmathsetmacro{\domainZMin}{1}
\pgfmathsetmacro{\domainZLength}{7}
\pgfmathsetmacro{\domainZMax}{\domainZMin+\domainZLength}
\pgfmathsetmacro{\domainDxOffset}{1}
\pgfmathsetmacro{\domainDzOffset}{1}
\definecolor{domain-color}{gray}{0.7}
\definecolor{scatterer-color}{gray}{0}

% Wavefront and time of flight
\definecolor{wavefront-color}{gray}{0.5}

% Axes
\newcommand*{\axesOffset}{1.5}

% Conics
\definecolor{conic-color}{gray}{0}
% 	Ellipse
% http://tex.stackexchange.com/questions/75017/draw-an-ellipse-from-the-two-focus-points-foci-and-the-sum-of-the-distances-fr
\newcommand*{\ellipsebyfoci}[4]{% options (draw, fill, color, etc.), focus pt1, focus pt2, cste
	\path[#1] let \p1=(#2), \p2=(#3), \p3=($(\p1)!.5!(\p2)$)
	in \pgfextra{
		\pgfmathsetmacro{\angle}{atan2(\x2-\x1,\y2-\y1)}
		\pgfmathsetmacro{\distfoci}{veclen(\x2-\x1,\y2-\y1)/1cm} % convert back to cm
		\pgfmathsetmacro{\axeone}{#4/2}
		\pgfmathsetmacro{\axetwo}{sqrt(\axeone^2 - (0.5*\distfoci)^2)}
%		\pgfmathsetmacro{\focal}{veclen(\x2-\x1,\y2-\y1)/2/1cm}
%		\pgfmathsetmacro{\lentotcm}{\focal*2*#4}
%		\pgfmathsetmacro{\axeone}{(\lentotcm - 2 * \focal)/2+\focal}
%		\pgfmathsetmacro{\axetwo}{sqrt((\lentotcm/2)*(\lentotcm/2)-\focal*\focal}
	}
%	(\p3) ellipse[x radius=\axeone cm,y radius=\axetwo cm, rotate=\angle];
	(\p3) ellipse[x radius=\axeone,y radius=\axetwo, rotate={90-\angle}];
%	(\p1) -- node[midway, sloped, above]{\distfoci} (\p2);
}