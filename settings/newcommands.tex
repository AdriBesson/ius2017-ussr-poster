%%%%%%%%%%%%%%%%%%%%%%%%%%%%%%%%%%%%%%%%%%%%%%%%%%%
% Signal Processing Laboratory (LTS5) - EPFL      %
% LaTeX beamposter template                       %
% Authors:                                        %
%   D. Perdios – dimitris.perdios@epfl.ch         %
%   A. Besson – adrien.besson@epfl.ch             %
% v0.1 - 08.05.17                                 %
% Typeset configuration: pdfLaTeX + Biber         %
%%%%%%%%%%%%%%%%%%%%%%%%%%%%%%%%%%%%%%%%%%%%%%%%%%%


% New command used to create a blank column

\newcommand{\AddBlankColumn}[1]{
	\begin{column}{#1}\end{column}
}

%%% SIZES OF THE MARGINS AND COLUMNS

%%% CUSTOM MAKETITLE COMMAND %%%
\newlength{\LogoWidthLeft}
\settowidth{\LogoWidthLeft}{\includegraphics[height=\LogoHeight]{logos/lts5_logo.pdf}}
\newlength{\LogoWidthRight}
\settowidth{\LogoWidthRight}{\includegraphics[height=\LogoHeight]{logos/epfl_logo.pdf}}
\setlength{\titlewidth}{\LeftColumnWidth + \RightColumnWidth + \BlankColumnWidthCenter- \LogoWidthLeft - \LogoWidthRight - \BlankColumnWidthLeft-\BlankColumnWidthRight}
\newsavebox{\titlebox}
\newlength{\myposterheight}
\renewcommand{\maketitle}{%
	\savebox{\titlebox}{%
		\hspace{\BlankColumnWidthLeft}
		\begin{minipage}{\LogoWidthLeft}%
			\includegraphics[height=\LogoHeight]{logos/basp_logo.png}
			\vspace{\LogoSkip}
			\centering
			\includegraphics[height=\LogoHeight]{logos/lts5_logo.pdf}
		\end{minipage} 
		\begin{minipage}{\titlewidth}%
			\centering
			\LARGE\inserttitle \leavevmode\\%
			\normalsize\insertauthor \leavevmode\\%
			\small\insertinstitute %
		\end{minipage} 
		\begin{minipage}{\LogoWidthRight}%
			\includegraphics[height=\LogoHeight]{logos/epfl_logo.pdf}
			\vspace{\LogoSkip}
			\centering
			\includegraphics[height=\LogoHeight]{logos/lts5_logo.pdf}
		\end{minipage}
		\hspace{\BlankColumnWidthRight}
	}%
	\settoheight{\titleheight}{\usebox{\titlebox}}
	\setlength{\myposterheight}{0.97\textheight - \titleheight - \TitleColumnSpace}
	\usebox{\titlebox}
}

% Typesetting
%\newcommand{\keywords}[1]{\noindent\textbf{Keywords:} #1}

% Math
\newcommand*{\abs}[1][]{\left\lvert#1\right\rVert}
\newcommand*{\norm}[2][]{\left\lVert#2\right\rVert_{#1}}
\newcommand*{\zeronorm}[1]{\norm[0]{#1}}
\newcommand*{\onenorm}[1]{\norm[1]{#1}}
\newcommand*{\twonorm}[1]{\norm[2]{#1}}
\newcommand*{\twoonenorm}[1]{\norm[2,1]{#1}}
\newcommand*{\fronorm}[1]{\norm[F]{#1}}
\newcommand*{\infnorm}[1]{\norm[\infty]{#1}}
\newcommand*{\pnorm}[1]{\norm[p]{#1}}
\newcommand*{\tvnorm}[1]{\norm[TV]{#1}}

% Sets
\newcommand*{\C}{\mathbb{C}}
\newcommand*{\R}{\mathbb{R}}
\newcommand*{\Q}{\mathbb{Q}}
\newcommand*{\Z}{\mathbb{Z}}
\newcommand*{\N}{\mathbb{N}}

% argmin, argmax, ...
\DeclareMathOperator*{\argmin}{argmin} % the star-version allows to have limit typeset bellow
\DeclareMathOperator*{\argmax}{argmax} % the star-version allows to have limit typeset bellow
% Matrix, vector, transpose, adjoint, ...
\newcommand*{\mat}[1]{\mathsf{#1}}
\newcommand*{\adjmat}[1]{\mathsf{#1}^T}
\renewcommand*{\vec}[1]{\bm{#1}}
\newcommand*{\tran}{^\mathsf{T}}
%\newcommand*{\tran}{^\intercal}
%\newcommand*{\tran}{^{\scriptscriptstyle\top}}
%\newcommand*{\tran}{\mkern-1mu{}_{}^{\scriptscriptstyle\top}\mkern-4mu}
\newcommand*{\conj}{^*}
\newcommand*{\adj}{^\dagger}
% Function, functional, ...
\newcommand*{\fun}[1]{#1}
\newcommand*{\funal}[1]{\mathcal{#1}}
% Transform, Dictionary, ...
\newcommand*{\dict}{\mat{D}}
\newcommand*{\transf}{\mat{\Psi}}
% Such that, subject to,...
\newcommand*{\suchthat}{\mid}
\newcommand*{\subjectto}{\,\,\mathrm{s.t.}\,\,}
\newcommand{\ser}[2]{#1^#2}

% US specific
\newcommand*{\medium}{\Omega}
\newcommand*{\transducer}{\Pi}
\newcommand*{\reflectivity}{\gamma}
\newcommand*{\rawdata}{m}
\newcommand*{\eaimpulseresp}{\fun{h_{el-ac}}}
\newcommand*{\excitation}{\fun{e}}
\newcommand*{\htx}{\fun{h_{Tx}}}
\newcommand*{\hrx}{\fun{h_{Rx}}}
\newcommand*{\pewavelet}{\fun{v_{pe}}}
\newcommand*{\eldir}{o}
\newcommand*{\wavelength}{\lambda}
\newcommand*{\noise}{n}

% Citations and referencs
%\newcommand*{\myciteauthor}[1]{\citeauthor{#1}~\cite{#1}}

% Latin expressions
% 	.\@ -> abbretiation period (for correct spacing)
\newcommand{\etc}{etc.\xspace}
\newcommand{\ie}{i.e.\@\xspace} % i.e. should NOT be italicized
\newcommand{\eg}{e.g.\@\xspace} % e.g. should NOT be italicized
\newcommand{\etal}{\textit{et al.\@}\xspace}
\newcommand*{\vs}{vs.\@\xspace}
\newcommand{\invitro}{\textit{in vitro}\xspace}
\newcommand{\Invitro}{\textit{In vitro}\xspace}
\newcommand{\invivo}{\textit{in vivo}\xspace}
\newcommand{\exvivo}{\textit{ex vivo}\xspace}
\newcommand{\Invivo}{\textit{In vivo}\xspace}

% Others
\newcommand{\xmark}{x\xspace}
\newcommand*{\tabhighlight}[1]{\textbf{#1}}
%\newcommand{\cmark}{\ding{51}}%
\newcommand*{\notavail}{\textcolor{red}{$\bigotimes$}}
%\newcommand{\notavail}{\textcolor{red}{\ding{53}}}%

% Specific abbreviations
\newcommand{\IIS}{ETHZ--IIS}
\newcommand{\LTS}{EPFL--LTS5}
\newcommand{\CSEM}{CSEM}
\newcommand{\USTOGO}{US2GO}
\newcommand{\nanotera}{Nano--Tera}
\newcommand{\Leleven}{L\num{11}-\num{4}v\xspace}
\newcommand{\Ltwelve}{L\num{12}-\num{5}~\num{50}mm\xspace}


